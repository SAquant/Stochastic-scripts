\documentclass[]{article}
\usepackage{lmodern}
\usepackage{amssymb,amsmath}
\usepackage{ifxetex,ifluatex}
\usepackage{fixltx2e} % provides \textsubscript
\ifnum 0\ifxetex 1\fi\ifluatex 1\fi=0 % if pdftex
  \usepackage[T1]{fontenc}
  \usepackage[utf8]{inputenc}
\else % if luatex or xelatex
  \ifxetex
    \usepackage{mathspec}
  \else
    \usepackage{fontspec}
  \fi
  \defaultfontfeatures{Ligatures=TeX,Scale=MatchLowercase}
\fi
% use upquote if available, for straight quotes in verbatim environments
\IfFileExists{upquote.sty}{\usepackage{upquote}}{}
% use microtype if available
\IfFileExists{microtype.sty}{%
\usepackage{microtype}
\UseMicrotypeSet[protrusion]{basicmath} % disable protrusion for tt fonts
}{}
\usepackage[margin=1in]{geometry}
\usepackage{hyperref}
\hypersetup{unicode=true,
            pdftitle={Problem12},
            pdfauthor={Mbongiseni Dlamini},
            pdfborder={0 0 0},
            breaklinks=true}
\urlstyle{same}  % don't use monospace font for urls
\usepackage{color}
\usepackage{fancyvrb}
\newcommand{\VerbBar}{|}
\newcommand{\VERB}{\Verb[commandchars=\\\{\}]}
\DefineVerbatimEnvironment{Highlighting}{Verbatim}{commandchars=\\\{\}}
% Add ',fontsize=\small' for more characters per line
\usepackage{framed}
\definecolor{shadecolor}{RGB}{248,248,248}
\newenvironment{Shaded}{\begin{snugshade}}{\end{snugshade}}
\newcommand{\AlertTok}[1]{\textcolor[rgb]{0.94,0.16,0.16}{#1}}
\newcommand{\AnnotationTok}[1]{\textcolor[rgb]{0.56,0.35,0.01}{\textbf{\textit{#1}}}}
\newcommand{\AttributeTok}[1]{\textcolor[rgb]{0.77,0.63,0.00}{#1}}
\newcommand{\BaseNTok}[1]{\textcolor[rgb]{0.00,0.00,0.81}{#1}}
\newcommand{\BuiltInTok}[1]{#1}
\newcommand{\CharTok}[1]{\textcolor[rgb]{0.31,0.60,0.02}{#1}}
\newcommand{\CommentTok}[1]{\textcolor[rgb]{0.56,0.35,0.01}{\textit{#1}}}
\newcommand{\CommentVarTok}[1]{\textcolor[rgb]{0.56,0.35,0.01}{\textbf{\textit{#1}}}}
\newcommand{\ConstantTok}[1]{\textcolor[rgb]{0.00,0.00,0.00}{#1}}
\newcommand{\ControlFlowTok}[1]{\textcolor[rgb]{0.13,0.29,0.53}{\textbf{#1}}}
\newcommand{\DataTypeTok}[1]{\textcolor[rgb]{0.13,0.29,0.53}{#1}}
\newcommand{\DecValTok}[1]{\textcolor[rgb]{0.00,0.00,0.81}{#1}}
\newcommand{\DocumentationTok}[1]{\textcolor[rgb]{0.56,0.35,0.01}{\textbf{\textit{#1}}}}
\newcommand{\ErrorTok}[1]{\textcolor[rgb]{0.64,0.00,0.00}{\textbf{#1}}}
\newcommand{\ExtensionTok}[1]{#1}
\newcommand{\FloatTok}[1]{\textcolor[rgb]{0.00,0.00,0.81}{#1}}
\newcommand{\FunctionTok}[1]{\textcolor[rgb]{0.00,0.00,0.00}{#1}}
\newcommand{\ImportTok}[1]{#1}
\newcommand{\InformationTok}[1]{\textcolor[rgb]{0.56,0.35,0.01}{\textbf{\textit{#1}}}}
\newcommand{\KeywordTok}[1]{\textcolor[rgb]{0.13,0.29,0.53}{\textbf{#1}}}
\newcommand{\NormalTok}[1]{#1}
\newcommand{\OperatorTok}[1]{\textcolor[rgb]{0.81,0.36,0.00}{\textbf{#1}}}
\newcommand{\OtherTok}[1]{\textcolor[rgb]{0.56,0.35,0.01}{#1}}
\newcommand{\PreprocessorTok}[1]{\textcolor[rgb]{0.56,0.35,0.01}{\textit{#1}}}
\newcommand{\RegionMarkerTok}[1]{#1}
\newcommand{\SpecialCharTok}[1]{\textcolor[rgb]{0.00,0.00,0.00}{#1}}
\newcommand{\SpecialStringTok}[1]{\textcolor[rgb]{0.31,0.60,0.02}{#1}}
\newcommand{\StringTok}[1]{\textcolor[rgb]{0.31,0.60,0.02}{#1}}
\newcommand{\VariableTok}[1]{\textcolor[rgb]{0.00,0.00,0.00}{#1}}
\newcommand{\VerbatimStringTok}[1]{\textcolor[rgb]{0.31,0.60,0.02}{#1}}
\newcommand{\WarningTok}[1]{\textcolor[rgb]{0.56,0.35,0.01}{\textbf{\textit{#1}}}}
\usepackage{graphicx,grffile}
\makeatletter
\def\maxwidth{\ifdim\Gin@nat@width>\linewidth\linewidth\else\Gin@nat@width\fi}
\def\maxheight{\ifdim\Gin@nat@height>\textheight\textheight\else\Gin@nat@height\fi}
\makeatother
% Scale images if necessary, so that they will not overflow the page
% margins by default, and it is still possible to overwrite the defaults
% using explicit options in \includegraphics[width, height, ...]{}
\setkeys{Gin}{width=\maxwidth,height=\maxheight,keepaspectratio}
\IfFileExists{parskip.sty}{%
\usepackage{parskip}
}{% else
\setlength{\parindent}{0pt}
\setlength{\parskip}{6pt plus 2pt minus 1pt}
}
\setlength{\emergencystretch}{3em}  % prevent overfull lines
\providecommand{\tightlist}{%
  \setlength{\itemsep}{0pt}\setlength{\parskip}{0pt}}
\setcounter{secnumdepth}{0}
% Redefines (sub)paragraphs to behave more like sections
\ifx\paragraph\undefined\else
\let\oldparagraph\paragraph
\renewcommand{\paragraph}[1]{\oldparagraph{#1}\mbox{}}
\fi
\ifx\subparagraph\undefined\else
\let\oldsubparagraph\subparagraph
\renewcommand{\subparagraph}[1]{\oldsubparagraph{#1}\mbox{}}
\fi

%%% Use protect on footnotes to avoid problems with footnotes in titles
\let\rmarkdownfootnote\footnote%
\def\footnote{\protect\rmarkdownfootnote}

%%% Change title format to be more compact
\usepackage{titling}

% Create subtitle command for use in maketitle
\providecommand{\subtitle}[1]{
  \posttitle{
    \begin{center}\large#1\end{center}
    }
}

\setlength{\droptitle}{-2em}

  \title{Problem12}
    \pretitle{\vspace{\droptitle}\centering\huge}
  \posttitle{\par}
    \author{Mbongiseni Dlamini}
    \preauthor{\centering\large\emph}
  \postauthor{\par}
      \predate{\centering\large\emph}
  \postdate{\par}
    \date{17/02/2020}


\begin{document}
\maketitle

\begin{Shaded}
\begin{Highlighting}[]
\CommentTok{#library(expm)}
\KeywordTok{library}\NormalTok{(Matrix)}
\KeywordTok{library}\NormalTok{(matlib)}
\end{Highlighting}
\end{Shaded}

\begin{verbatim}
## Warning: package 'matlib' was built under R version 3.6.2
\end{verbatim}

\begin{Shaded}
\begin{Highlighting}[]
\NormalTok{N <-}\StringTok{ }\DecValTok{50} \CommentTok{#POPULATI0N SIZE}
\NormalTok{nrows <-}\StringTok{ }\NormalTok{N }\CommentTok{#select number of rows}
\NormalTok{ncols <-}\StringTok{ }\NormalTok{N }\CommentTok{#number of columns}
\NormalTok{product <-}\StringTok{ }\NormalTok{nrows}\OperatorTok{*}\NormalTok{ncols}
\NormalTok{vec=}\StringTok{ }\KeywordTok{rep}\NormalTok{(}\DecValTok{0}\NormalTok{,product)}


\NormalTok{B <-}\StringTok{ }\FloatTok{0.05} \CommentTok{#CONATACT RATE}
\NormalTok{rr <-}\StringTok{ }\DecValTok{5}\OperatorTok{/}\DecValTok{22} \CommentTok{# Recovery rate}
\NormalTok{br_dr <-}\StringTok{ }\DecValTok{5}\OperatorTok{/}\DecValTok{22} \CommentTok{# Birth rate = Death rate}
\NormalTok{n <-}\StringTok{ }\DecValTok{0}

\CommentTok{# FUCTION new infections}

\NormalTok{b <-}\StringTok{ }\ControlFlowTok{function}\NormalTok{(x)\{}
\NormalTok{  new_inf <-}\StringTok{ }\NormalTok{x}\OperatorTok{*}\NormalTok{(N}\OperatorTok{-}\NormalTok{x)}\OperatorTok{*}\NormalTok{((B)}\OperatorTok{/}\NormalTok{N)}
  \KeywordTok{return}\NormalTok{(new_inf)}
\NormalTok{\}}
\CommentTok{# FUCTION new deaths}

\NormalTok{d <-}\StringTok{ }\ControlFlowTok{function}\NormalTok{(x)\{}
\NormalTok{  new_deaths <-}\StringTok{ }\NormalTok{(}\DecValTok{2}\OperatorTok{*}\NormalTok{rr)}\OperatorTok{*}\NormalTok{x}
  \KeywordTok{return}\NormalTok{(new_deaths)}
\NormalTok{\}}
\CommentTok{#Make matrix}
\ControlFlowTok{for}\NormalTok{(row }\ControlFlowTok{in} \DecValTok{1}\OperatorTok{:}\NormalTok{nrows)}
\NormalTok{\{}
  \ControlFlowTok{for}\NormalTok{ (column }\ControlFlowTok{in} \DecValTok{1}\OperatorTok{:}\NormalTok{ncols)\{}
    
    \ControlFlowTok{if}\NormalTok{ (row}\OperatorTok{==}\DecValTok{1} \OperatorTok{&&}\StringTok{ }\NormalTok{column }\OperatorTok{==}\StringTok{ }\DecValTok{1}\NormalTok{)\{}
\NormalTok{      vec[column}\OperatorTok{+}\NormalTok{n] <-}\StringTok{ }\OperatorTok{-}\NormalTok{(}\KeywordTok{b}\NormalTok{(row)}\OperatorTok{+}\KeywordTok{d}\NormalTok{(row))}
\NormalTok{    \}}
    \ControlFlowTok{else} \ControlFlowTok{if}\NormalTok{ (row}\OperatorTok{==}\DecValTok{1} \OperatorTok{&&}\StringTok{ }\NormalTok{column }\OperatorTok{==}\StringTok{ }\DecValTok{2}\NormalTok{)\{}
\NormalTok{      vec[column}\OperatorTok{+}\NormalTok{n] <-}\StringTok{ }\KeywordTok{b}\NormalTok{(row)}
\NormalTok{    \}}
    
    \ControlFlowTok{else} \ControlFlowTok{if}\NormalTok{(row }\OperatorTok{-}\StringTok{ }\NormalTok{column }\OperatorTok{==}\StringTok{ }\DecValTok{1}\NormalTok{)}
\NormalTok{    \{}
\NormalTok{      vec[column}\OperatorTok{+}\NormalTok{n] <-}\StringTok{ }\KeywordTok{d}\NormalTok{(row)}
\NormalTok{    \}}
    \ControlFlowTok{else} \ControlFlowTok{if}\NormalTok{(row}\OperatorTok{==}\NormalTok{column)}
\NormalTok{    \{}
\NormalTok{      vec[column}\OperatorTok{+}\NormalTok{n] <-}\StringTok{ }\OperatorTok{-}\NormalTok{(}\KeywordTok{d}\NormalTok{(row)}\OperatorTok{+}\KeywordTok{b}\NormalTok{(row))}
\NormalTok{    \}}
    \ControlFlowTok{else} \ControlFlowTok{if}\NormalTok{(row}\OperatorTok{-}\NormalTok{column }\OperatorTok{==}\StringTok{ }\DecValTok{-1}\NormalTok{)}
\NormalTok{    \{}
\NormalTok{      vec[column}\OperatorTok{+}\NormalTok{n] <-}\StringTok{ }\KeywordTok{b}\NormalTok{(row)}
\NormalTok{    \}}
    
\NormalTok{  \}}
\NormalTok{    n <-}\StringTok{ }\NormalTok{n}\OperatorTok{+}\NormalTok{ncols}
\NormalTok{\}}
\NormalTok{    D <-}\StringTok{ }\KeywordTok{matrix}\NormalTok{(vec, }\DataTypeTok{nrow =}\NormalTok{ nrows,}\DataTypeTok{ncol =}\NormalTok{ ncols, }\DataTypeTok{byrow =} \OtherTok{TRUE}\NormalTok{)}
\KeywordTok{View}\NormalTok{(D)}

\CommentTok{### The computation}
\NormalTok{one <-}\StringTok{ }\KeywordTok{rep}\NormalTok{(}\DecValTok{1}\NormalTok{,N)}
\NormalTok{ONE <-}\StringTok{ }\KeywordTok{matrix}\NormalTok{(one)}
\KeywordTok{View}\NormalTok{(ONE)}
\NormalTok{inverse_D <-}\StringTok{ }\DecValTok{-1}\OperatorTok{*}\KeywordTok{inv}\NormalTok{(D)}
\KeywordTok{View}\NormalTok{(inverse_D)}
\NormalTok{avg_times <-}\StringTok{ }\NormalTok{inverse_D }\OperatorTok\StringTok{ }\NormalTok{ONE}
\NormalTok{I <-}\StringTok{ }\NormalTok{D }\OperatorTok\StringTok{ }\NormalTok{inverse_D}
\KeywordTok{View}\NormalTok{(I)}
\KeywordTok{View}\NormalTok{(avg_times)}
\CommentTok{## part b}

\CommentTok{# Here we solve the folowing equation:}
\NormalTok{second_moment <-}\StringTok{ }\NormalTok{inverse_D }\OperatorTok\StringTok{ }\NormalTok{avg_times}
\NormalTok{variance <-}\StringTok{ }\NormalTok{second_moment }\OperatorTok{-}\StringTok{ }\NormalTok{(avg_times}\OperatorTok{^}\DecValTok{2}\NormalTok{)}
\KeywordTok{View}\NormalTok{(variance)}
\NormalTok{R_}\DecValTok{0}\NormalTok{ <-}\StringTok{ }\NormalTok{B}\OperatorTok{/}\NormalTok{(rr}\OperatorTok{+}\NormalTok{br_dr)}
\NormalTok{R_}\DecValTok{0}
\end{Highlighting}
\end{Shaded}

\begin{verbatim}
## [1] 0.11
\end{verbatim}

\begin{Shaded}
\begin{Highlighting}[]
\KeywordTok{plot}\NormalTok{(avg_times, }\DataTypeTok{xlab =} \StringTok{"Initial Number of Infecteds"}\NormalTok{, }\DataTypeTok{ylab =} \StringTok{"Time to Aborption"}\NormalTok{)}
\end{Highlighting}
\end{Shaded}

\includegraphics{problem12_files/figure-latex/unnamed-chunk-1-1.pdf}


\end{document}
